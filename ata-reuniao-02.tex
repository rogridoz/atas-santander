
\documentclass[a4paper, 11pt]{article}
\usepackage{comment} % enables the use of multi-line comments (\ifx \fi) 
\usepackage{fullpage} % changes the margin
\usepackage[swedish]{babel}
\usepackage[utf8]{inputenc}
\usepackage{graphicx}
\usepackage{multicol}
\usepackage{float}
\usepackage{fancyhdr}
\usepackage{enumitem}
\pagestyle{fancy} 
\usepackage{pdfpages}
%\usepackage[head=128pt]{geometry}
\title{Santander Brasil}
\author{Rodrigo dos Santos}
\usepackage{geometry}
\setlength{\footskip}{0.1pt}
\setlength{\headheight}{80pt}
\setlength{\topmargin}{0pt}
\setlength\parindent{0pt}
\fancypagestyle{style1}{
\lhead{\includegraphics[width=8cm]{Santander-Logo.png}}
% \rhead{Santander 20xx-xx-xx\\
% Sede\\
% Av Juscelino K \\}
% \renewcommand{\headrulewidth}{0pt}
\cfoot{
\makebox{}\\
\makebox{}\\
\makebox[0.1\linewidth]{\rule{0.1\linewidth}{0.1pt}} \hspace{1cm} \makebox[0.1\linewidth]{\rule{0.1\linewidth}{0.1pt}} \hspace{1cm} \makebox[0.1\linewidth]{\rule{0.1\linewidth}{0.1pt}} \hspace{1cm} \makebox[0.1\linewidth]{\rule{0.1\linewidth}{0.1pt}} \hspace{1cm}\\}
}
\fancypagestyle{style2}{
\lhead{\includegraphics[width=7cm]{Santander-Logo.png}}
\rhead{Santander 20xx-xx-xx\\
Sede\
Av Juscelino K\\}
\renewcommand{\headrulewidth}{0pt}
\cfoot{}
}
\begin{document}
\pagestyle{style1}

\textbf{Data: 31/01/2022} % inserir data aqui

\textbf{Local: P23-R04} % Definir local da reunião 

\textbf{Participantes:} 
\begin{description}
\item (PP) Vitória Marine
\item (SP) Rodrigo dos Santos

\end{description}

\makebox[\linewidth]{\rule{\linewidth}{0.4pt}}\\
\textbf{Pessoa responsável dá início à reunião pelas 10h, referente ao alinhamento sobre a fila de avaliação de propostas do Caminho do Empreendedor:} 
\begin{enumerate}

\item Apresentação de report desenvolvido semanalmente para avaliação de propostas


\item Análise de retorno relativo à descrição dos filtros combo


\item Reavaliação da rotina programada




\end{enumerate}
\makebox[\linewidth]{\rule{\linewidth}{0.4pt}}\\

\section*{1.}
Ênfase de apresentação em propostas negadas e devolvidas, devidamente graficadas em forma de funil, onde foi possível demonstrar e questionar os filtros de retorno e suas respectivas dimensões.

\section*{2.}
Clientes enquadrados sob o filtro "Sem Porte" são aqueles dos quais seu faturamento não está de acordo com o determinado na categoria de microempreendedor individual. Faz-se necessário reavaliar a base disponibilizada para a submissão de propostas via robô. Provável solução observada: coluna de faturamento deve ser referente à sintese.

Clientes enquadrados sob o filtro "Produto Inadequado" são aqueles dos quais já possuem produto com o banco e realizaram solicitação de um novo sem a devida necessidade, conforme julgado. Compete assim reavaliar como aditamento.

Clientes enquadrados sob o filtro "Dados desatualizados" possuem informações sensíveis incorretas, como faturamento nulo. Análogo ao anterior, deve-se reenviar como aditamento de produto, ao invés de concessão.


\section*{3. }
Após breve análise sobre propostas negadas e/ou devolvidas, faz-se necessário reordenar o tipo de operação de concessão para aditamento, se assim julgado.





\section*{3.}

 
 \section*{}
\textbf{Pessoa responsável dá por terminada a reunião às 11:10 de dia 31/01/2022.}

\setlength{\textwidth}{14cm}\textbf{Glossário}
\textbf{Aditamento:} Correção de produtos e/ou suas características.
\textbf{Log:} Registro automático de operações em forma de arquivo de texto

%     \section*{§23. Mötets avslutande}
% Elin Luedtke avslutar mötet kl 16.34..
% \thispagestyle{style2}
% \makebox{}\\
% \makebox{}\\



% 1.alinhamentos faturamentos - big data - query - KOLB
% 2. jornada - aditamento - acesso alguma base de acordo - na sintese
% 3. parecer final - confirmar com o Ale
% sem porte - já tem limite







% \makebox[0.4\linewidth]{\rule{0.4\linewidth}{0.4pt}} \hspace{1cm} \makebox[0.4\linewidth]{\rule{0.4\linewidth}{0.4pt}} \hspace{1cm}\\
% \makebox[0.4\linewidth]{Sek-namn} \hspace{1cm}
% \makebox[0.4\linewidth]{Ordf-namn} \hspace{1cm}\\
% \makebox[0.4\linewidth]{Mötessekreterare} \hspace{1cm}
% \makebox[0.4\linewidth]{Mötesordförande} \hspace{1cm}\\
% \makebox{}\\
% \makebox{}\\
% \makebox{}\\
% \noindent\makebox[0.4\linewidth]{\rule{0.4\linewidth}{0.4pt}} \hspace{1cm} \makebox[0.4\linewidth]{\rule{0.4\linewidth}{0.4pt}} \hspace{1cm}\\
% \makebox[0.4\linewidth]{Justerare 1} \hspace{1cm}
% \makebox[0.4\linewidth]{Justerare 2} \hspace{1cm}\\
% \makebox[0.4\linewidth]{Justerare} \hspace{1cm}
% \makebox[0.4\linewidth]{Justerare} \hspace{1cm}\\



\end{document}
